% \iffalse meta-comment
%
%% File: unicodefonttable.dtx (C) Copyright 2019-2020 Frank Mittelbach
%
% It may be distributed and/or modified under the conditions of the
% LaTeX Project Public License (LPPL), either version 1.3c of this
% license or (at your option) any later version.  The latest version
% of this license is in the file
%
%    https://www.latex-project.org/lppl.txt
%
%
% The development version of the bundle can be found below
%
%    https://github.com/FrankMittelbach/fmitex/
%
% for those people who are interested or want to report an issue.
%

\def\unicodefonttabledate   {2021/09/12}
\def\unicodefonttableversion{v0.9d}


%<*driver>
\documentclass{l3doc}
\usepackage{unicodefonttable}

% bug fix for l3doc.cls
\ExplSyntaxOn
\cs_set_protected:Npn \__codedoc_macro_typeset_one:nN #1#2
  {
    \vbox_set:Nn \l__codedoc_macro_box
      {
        \vbox_unpack_drop:N \l__codedoc_macro_box
        \hbox { \llap { \__codedoc_print_macroname:nN {#1} #2
            \MacroFont       % <----- without it the \ is in lmr10 if a link is made
            \      
        } }
      }
    \int_incr:N \l__codedoc_macro_int
  }
\ExplSyntaxOff

\EnableCrossrefs
\CodelineIndex
\begin{document}
  \DocInput{unicodefonttable.dtx}
\end{document}
%</driver>
%
% \fi
%
%
% \newcommand\key [1]{\texttt{#1}}
% \newcommand\kval[1]{\texttt{#1}}
%
% \newcommand\keysetup[1]{\noindent\marginpar{\raggedleft Key setup: #1}\ignorespaces}
%
%
% \title{The \texttt{unicodefonttable} package\thanks{}}
% \author{Frank Mittelbach}
%
% \maketitle
%
%
% \begin{abstract}
%   A package for typesetting font tables for larger fonts, e.g.,
%    TrueType or OpenType Unicode fonts.
% \end{abstract}
%
%
% \tableofcontents
%
%
%
% \section{Introduction}
%
%    When I started to write a new chapter for the third edition of
%    \emph{The \LaTeX{} Companion} on modern fonts available for
%    different \LaTeX{} engines, I was a bit surprised that I couldn't
%    find a way to easily typeset tables showing the glyphs available
%    TrueType or OpenType fonts. The \pkg{nfssfont} package available
%    with \LaTeX{} only supports fonts from the 8-bit world, but
%    modern fonts that can be used with \XeTeX{} or \LuaTeX{} can
%    contain thousands of glyphs and having a method to display what
%    is available in them was rather important for me.
%
%    I therefore set out to write my own little package and what has
%    started as an afternoon exercise ended up being this package
%    offering more and more bells and whistles for typesetting such
%    font tables.
%
%    As there can be many glyphs in such fonts a tabular
%    representation of them might run for several pages, so the
%    package internally uses the \pkg{longtable} package to handle
%    that.
%
%    In most cases the glyphs inside the fonts are index by their
%    Unicode numbers so it is natural to display them sorted by their
%    position in the Unicode character set.
%
%    Unicode is organized in named blocks such as ``Basic Latin'',
%    ``Latin-1~Supplement'', etc.\ typically consisting of 265
%    characters each.\footnote{Some blocks are smaller and those
%    containing the Asian ideographs are much larger.} It is therefore
%    helpful to use these block names as sub-titles within the table
%    in order to easily find the information one is looking for.
%
%    A common way to represent the number of a single Unicode
%    character is \texttt{U+} followed by four (or more)
%    hexadecimals. For example \texttt{U+0041} represents the letter
%    ``A'' and \texttt{U+20AC} the Euro currency symbol ``\texteuro''.
%    We use this convention by showing a Unicode range (of sixteen
%    character) at the left of each table row, e.g., \texttt{U+0040 -
%    004F}, followed by the sixteen glyphs in the range. Thus that
%    particular table row from the ``Basic Latin'' block would show
%    something like
%\begin{quote}
%  \displayfonttable*[noheader,range-start=0040, range-end=004F]{Latin Modern Sans}
%\end{quote}
%    If a Unicode character has no glyph representation in a given
%    font then this is indicated by a special symbol (by default a
%    colored hyphen).
%
%    In order to easily find any Unicode character the table shows by
%    default sixteen hex digits a column headings so in order to find
%    Euro currency symbol (\texttt{U+20AC}) one first has to find the
%    right row which is the range \texttt{U+20A0 - 20AF} and then find
%    the \texttt{C} column in the row and there it is (or an
%    indication that the font is missing the glyph you are looking for
%    --- the line shows that for other slots).
%
%\begin{quote}
%  \displayfonttable*[noheader,range-start=20A0, range-end=20AF]{Latin Modern Sans}
%\end{quote}
%
%    It is sometimes useful to compare two fonts with each other by
%    filling the table with glyphs from a secondary font if the
%    primary font is missing them. For example, the next display shows
%    two rows of Latin Modern Math (black glyphs) and instead of showing
%    a missing glyph symbol in most slots, we rope in glyphs from New
%    Computer Modern Math, which has a much larger glyph set (red
%    glyphs with gray background).
%
%\begin{quote}
%\displayfonttable[noheader,nostatistics,display-block=none,
%   compare-with=NewCMMath-Regular.otf,range-start=2A00,range-end=2A1F]
%                 {latinmodern-math.otf}
%\end{quote}
%
%
% \section{The user interface}
%
%    The package offers one command to typeset a font table. The
%    appearance of the table can be customized by specifying key/value
%    pairs.
%
% \begin{function}{\displayfonttable}
%   \begin{syntax}
%     \cs{displayfonttable} \texttt{*} \oarg{key/value-list} \Arg{font-name} \oarg{font-features}
%   \end{syntax}
%    The \meta{font-name} is the font to be displayed. This and the
%    \meta{font-features} argument are passed to \pkg{fontspec}, thus they
%    should follow the conventions of that package for specifying a
%    font. The \meta{key/value-list} offers customization
%    possibilities discussed below.
%
%    The \cs{displayfonttable*} is a variant of the command, intended
%    for use with 8-bit legacy fonts. It presets some keys but
%    otherwise behaves identically.  The values used in that case are
%\begin{verbatim}
%     nostatistics, display-block=none, hex-digits=head, range-end=FF
%\end{verbatim}
%    For details see the next section.
% \end{function}
%

%

%
% \begin{function}{\fonttablesetup}
%   \begin{syntax}
%     \cs{fonttablesetup} \Arg{key/value-list}
%   \end{syntax}
%    Instead or in addition to specifying key/values to
%    \cs{displayfonttable} it is possible to set them up as
%    default. Inside \cs{displayfonttable} the defaults are applied first,
%    so that one can still overwrite their values on individual tables.
% \end{function}
%
%
% \begin{function}{\fonttableglyphcount}
%   \begin{syntax}
%     \cs{fonttableglyphcount}
%   \end{syntax}
%    While typesetting a font table the package keeps track of the
%    number of glyphs it find in the font. After the table has finished,
%    this value is available in \cs{fonttableglyphcount} and it is, for
%    example, used when statistics are produced. At the start of the
%    next table it is reset to zero.
% \end{function}
%
%
%
%
% \subsection{Keys and their values}
%
%    Several of the available keys are boolean accepting \texttt{true}
%    or \texttt{false} and usually exist in pairs so that one can
%    specify the desired behavior without the need to provide a value,
%    e.g., specifying \key{header} is equivalent to specifying
%    \key{header}\texttt{=true} or \key{noheader}\texttt{=false}, etc.
%
%    In the lists below the default settings are indicated by an
%    underline.
%
%  \begin{variable}{
%     header, noheader,
%     title-format, title-format-const,
%  }
%    The first set of keys are concerned with the overall look and
%    feel of the generated table.
% \begin{description}
% \item[\underline{\key{header}}, \key{noheader}]
%    These keys decide whether or not a header to the table is
%    produced.
%
% \item[\key{title-format}, \key{title-format-cont}]
%    These keys define what is provided as a header title or
%    continuation title if the table consists of several pages.
%    They expect code as their value. This code can contain \verb=#1=
%    and \verb=#2= to denote the \meta{font-name} and
%    \meta{font-features} arguments, respectively.
%
%    By default a title using the \cs{caption} command is produced; on
%    continuation titles without showing the \meta{font-features}.
%
% \end{description}
%
% \end{variable}
%
%
%  \begin{variable}{
%     display-block,
%     hex-digits, color,
%  }
%    These keys handle the inner parts of the table.
%
% \begin{description}
% \item[\key{display-block}]
%    The Unicode dataset is organized in named blocks typically 128 or
%    256 characters though some are noticeably larger and a few are
%    smaller. With the help of the \key{display-block} key it is
%    possible to specify if and how such blocks should be made visible.
%    The following values are supported:
%    \begin{description}
%    \item[\underline{\kval{titles}}]
%      Above each display block that contains glyphs the Unicode title
%      of the block is displayed.
%
%    \item[\kval{rules}]
%      Display blocks are only indicated by a \cs{midrule}.
%
%    \item[\kval{none}]
%      Display blocks are not indicated at all.
%    \end{description}
%
% \item[\key{hex-digits}]
%    To ease reading the table rows of hex digits are added to
%    it. Where or if this happens is controlled by this key. Allowed
%    values for it are the following:
%    \begin{description}
%    \item[\kval{block}]
%      A row of hex digits is placed at beginning of each Unicode
%      block containing glyphs in the displayed font.
%
%    \item[\kval{foot}]
%      A row is added to the foot of each table page.
%
%    \item[\underline{\kval{head}}]
%      A row is added to the top of each table page.
%
%    \item[\kval{head+foot}]
%      A row is added to the top and the foot of each table page.
%
%    \item[\kval{none}]
%      A hex digit rows are suppressed.
%    \end{description}%
%
% \item[\key{hex-digits-font}]
%    The font to use for the hex digits by default 
%    \underline{\cs{ttfamily}\cs{scriptsize}}.
%
% \item[\key{color}]
%    This key determines the color for parts of the table (hex digits
%    and Unicode ranges). It can be either \kval{none} or a color
%    specification as understood by the \cs{color} command.
%    The default is \underline{\texttt{blue}}.
%
% \end{description}
%
% \end{variable}
%
%
%  \begin{variable}{
%     statistics, nostatistics,
%     statistics-font,statistics-format,
%  }
%    The next set of keys allows altering the statistics that are produced.
% \begin{description}
%
% \item[\underline{\key{statistics}}, \key{nostatistics}]
%    These keys decide whether or not some statistics are listed at
%    the end of the table.
%
% \item[\key{statistics-font}]
%    The font used to typeset the statistics, default is \underline{\cs{normalfont}}.
%
%
% \item[\key{statistics-format}]
%    Code (text) to specify what should be typeset in the
%    statistics. One can use \verb=#1= for the \meta{font-name} and
%    \verb=#2= for the glyph count.
%    The material is typeset on a single line at the end of the
%    table. If several lines are needed you need to use  \cs{parbox}
%    or a similar construct.
%    \end{description}
% \end{variable}
%
%
%
%  \begin{variable}{
%     glyph-width,
%     missing-glyph, missing-glyph-font,  missing-glyph-color, 
%  }
%    Another set of keys deals customization on the glyph level.
% \begin{description}
% \item[\key{glyph-width}]
%    All glyphs are typeset in a box with the same width, the default
%    value is \underline{\texttt{6pt}} which is suitable for most 10pt
%    fonts 
%    and make the table fit comfortably into the text width of
%    a typical document.
%
% \item[\key{missing-glyph}]
%    If a slot in a row doesn't have a glyph in the font you may still
%    want display something to indicate this state. By giving the key
%    a value any arbitrary glyph or material can be typeset. The
%    default is to typeset a \underline{\kval{-}} in a special color.
%
%    Rows that contain no glyph whatsoever are not displayed at
%    all. Instead a small vertical space is added to indicate the one
%    or more rows are omitted.
%
% \item[\key{missing-glyph-font}]
%
%    The font used for the missing glyphs (default
%    \underline{\cs{ttfamily}\cs{scriptsize}}).
%
% \item[\key{missing-glyph-color}]
%    If not specified it uses the value specified with the \cs{color}
%    key. If you want a different color, e.g., \texttt{red}, you can
%    use a color value or you can specify \kval{none} to use no coloring.
%
% \end{description}
%
% \end{variable}
%
%
%  \begin{variable}{
%     compare-with,compare-color, compare-bgcolor
%  }
%
%    You can make comparisons between two fonts, which is useful, for
%    example when dealing with incomplete math fonts and you need to
%    see how well the symbols from one font blend with the supplement
%    symbols from another font.
%
% \begin{description}
% \item[\key{compare-color}, \key{compare-bgcolor}]
%
%    If given, the value is a \meta{comparison-font-name} that is used
%    to supply missing glyphs. This means that if the \meta{font-name}
%    to be displayed is missing a glyph in a slot, then the
%    \meta{comparison-font-name} is checked and it that font has the
%    glyph in question, it will be displayed instead of showing a
%    missing glyph indicator.
%
%    To distinguish real glyphs from missing but shown glyphs, they are
%    typically drawn in a different color and/or they get their
%    background colored.
%
% \item[\key{compare-with}]
%
%    Such substituted glyphs can be colored specially (default
%    \underline{\texttt{red}}) and you can have their background colored
%    (default is \underline{\texttt{black!10}}, i.e., a light gray).
%
% \end{description}
% \end{variable}
%
%
%  \begin{variable}{
%     range-start, range-end,
%  }
%    Finally there are two keys for restricting the display range.
% \begin{description}
% \item[\key{range-start}, \key{range-end}]
%    The full Unicode set of characters is huge  and checking every
%    slot to see if the current font contains a glyph in the slot
%    takes a long time and if you know that font contains only a
%    certain subset then you can speed up the table generation
%    considerably by limiting the search (and consequently the output
%    generation).
%    The \key{range-start} specifies where to start with the search
%    (default \underline{\texttt{0000}}) and \key{range-stop} gives
%    the last slot that is tested (default
%    \underline{\texttt{FFFF}}).\footnotemark
%
%    This is also quite useful in combination with the previous
%    \key{compare-with} key, to just display, for example, the greek
%    letters and see how glyphs from two fonts blend with each other.

% \end{description}
% \end{variable}\footnotetext{This is the whole of what
%    the \pkg{unicodefonttable} is currently able to handle, so right
%    now you can only reduce not extend!}
%
%
% \section{Example usage}
%
% In this section we show the results of a few calls to \cs{displayfonttable}.
%
% \subsection{Computer Modern Sans --- 8-bit font}
%
% Command used: \verb/\displayfonttable*[color=none, range-end=7F]{cmss10}/
% 
%\iftrue
%  \displayfonttable*[color=none, range-end=7F]{cmss10}
%\fi
%
% \subsection{TeX Gyre Heros (Helvetica)  --- 8-bit font}
% 
% Command used:
%\begin{verbatim}
% \displayfonttable*[color=red,nostatistics=false,
%                    hex-digits = head+foot,range-end = FF]{ec-qhvr}
%\end{verbatim}
% 
%\iftrue
% \displayfonttable*[color=red,nostatistics=false,
%                    hex-digits = head+foot,range-end = FF]{ec-qhvr}
%\fi 
% 
% \iffalse
% \subsection{Latin Modern Sans --- OTF font}
% 
% Command used:
%\begin{verbatim}
% \displayfonttable[hex-digits = block,
%                   title-cont-format = \caption[]{\emph{continued}}
%                  ]{Latin Modern Sans}
%\end{verbatim}
% 
%\iftrue
% \displayfonttable[hex-digits = block,
%                   title-cont-format = \caption[]{\emph{continued}}
%                  ]{Latin Modern Sans}
%\fi 
%
% \subsection{Latin Modern Math compared to New Computer Modern Math
%    in the range 2A00--2AFF}
% 
% Command used:
%\begin{verbatim}
%\displayfonttable[noheader,nostatistics,hex-digits = block,
%   compare-with=NewCMMath-Regular.otf,range-start=2A00,range-end=2AFF]
%                 {latinmodern-math.otf}
%\end{verbatim}
% 
%\iftrue
% \newpage
%\displayfonttable[noheader,nostatistics,hex-digits = block,
%   compare-with=NewCMMath-Regular.otf,range-start=2A00,range-end=2AFF]
%                 {latinmodern-math.otf}
%\fi 
% \fi
%
%\iffalse
%\section{\TeX{} Gyre Pagella (Palatino) oldstyle figures --- OTF font}
%
%\displayfonttable{TeX Gyre Pagella}[Numbers=OldStyle]
%\fi

%
% \StopEventually{\setlength\IndexMin{200pt}  \PrintIndex  }
%
%
% \section{The Implementation}
%
% \subsection{The package}
%
%    \begin{macrocode}
%<*package>
%    \end{macrocode}
%
%    By default the package uses coloring to improve the table
%    appearance and therefore requires a color package.
%    \begin{macrocode}
\RequirePackage{xcolor}
%    \end{macrocode}
%

%    \begin{macrocode}
%<@@=fmuft>
%    \end{macrocode}
%
%    We need the package \pkg{xparse} for specifying the document-level
%    interface commands and \pkg{l3keys2e} to use the \pkg{expl3} key
%    value methods within \LaTeXe{}. These packages automatically
%    require \pkg{expl3} so there is no need to load that explicitly.
%    Actually, \pkg{expl3}, \pkg{l3keys2e} and the \pkg{xparse}
%    functionality is now all part of the \LaTeX{} kernel so the next
%    line is actually not needed at all with a current \LaTeX{} kernel, but
%    in order to support older installations we keep it for now.
%
%    \begin{macrocode}
\RequirePackage{xparse,l3keys2e}
%    \end{macrocode}
%
% Here we introduce the package and specify its version number:
%    \begin{macrocode}
\ProvidesExplPackage{unicodefonttable}
                    {\unicodefonttabledate}
                    {\unicodefonttableversion}
                    {Producing font tables for Unicode and other fonts}
%    \end{macrocode}
%
%  \newcommand\hex[1]{$\langle\textit{hex}_{#1}\rangle$}
%
%
%
%
%                    
% \subsubsection{User interface commands}
%
%
%  Throughout the implementation we will define a number of keys (and
%  their allowed values). We introduce them at the point where they are
%    used, so they are sprinkled across the code.\footnote{This fits
%    with the way this package was developed. I first implemented a
%    single rigid table layout without configuration possibilities and
%    then thought about which parts I wanted to have flexible. I then
%    replaced the rigid code with code that is affected by setting
%    key/value pairs.}

%  \begin{macro}{\fonttablesetup}
%    To set up user defaults for the keys we provide a standard
%    interface. The command \cs{unicodefonttabletablesetup} expects a
%    key/value list and can be called as often as necessary.
%    \begin{macrocode}
\NewDocumentCommand \fonttablesetup { m }
  { \keys_set:nn {@@} {#1} \ignorespaces }
%    \end{macrocode}
%  \end{macro}
%
%
%
%  \begin{macro}{\displayfonttable}
%    The document-level command for generating a font table.
%    \begin{macrocode}
\NewDocumentCommand\displayfonttable {s O{} m o}{%
  \IfBooleanTF #1
     {
%    \end{macrocode}
%    For the starred form we preset a number of keys with values
%    suitable when displaying 8-bit legacy fonts.
%    With such fonts Unicode block headers make little
%    sense (as the fonts do not conform to Unicode font layout and
%    since they have at most 265 glyphs). It is therefore also unnecessary to
%    loop over the whole Unicode range of the first plane.
%    If necessary all of them can still be overwritten in the optional argument.       
%    \begin{macrocode}
       \@@_display_fonttable:nnn
          {nostatistics,display-block=none,hex-digits=head,range-end=FF,#2}
          {#3}{#4}
     }
     {
       \@@_display_fonttable:nnn {#2}{#3}{#4}
     }
}
%    \end{macrocode}
%  \end{macro}
%  
%  
%  \begin{macro}{\@@_display_fonttable:nnn}
%    This command is the main work horse of the
%    package. It produces a \texttt{longtable} containing all font
%    glyphs with 16 glyphs per row. The first optional argument is
%    used to configure the table through key value pairs, the
%    mandatory argument is the font name to display (in
%    \texttt{fontspec} conventions) and the final optional argument is
%    the font feature list if any. If the latter is not provided it
%    will get a special value (\texttt{--NoValue--}) assigned by
%    \texttt{xparse}, which is something that can be tested for.
%    \begin{macrocode}
\cs_new:Npn \@@_display_fonttable:nnn #1#2#3 {
  \group_begin:
%    \end{macrocode}
%    First initialize the font that should be displayed (perhaps with a
%    feature list) and then update the key value list using \verb=#1=. 
%    \begin{macrocode}
    \fontspec{#2}[#3]
    \keys_set:nn{@@}{#1}
%    \end{macrocode}
%    If the user has asked for a comparsion to some other font we need to set this up:
%    \begin{macrocode}
    \tl_if_empty:NF \l_@@_compare_with_tl
       {
         \setfontface \l_@@_compare_font_tl {\l_@@_compare_with_tl}[]
         \cs_set_eq:NN \@@_handle_missing_glyph:n
                       \@@_handle_missing_glyph_compare:n
       }
%    \end{macrocode}
%    The we start the table with 17 columns.
%    \begin{macrocode}
    \begin{longtable}{@{}c@{\quad}*{16}{c}@{}}
%    \end{macrocode}
%    Special headers and footers are set up first:
%    \begin{macrocode}
      \@@_setup_header_footer:nn{#2}{#3}
%    \end{macrocode}
%    Then we produce all table rows with the glyphs.
%    \begin{macrocode}
      \@@_produce_table_rows:
%    \end{macrocode}
%    At the very end we may typeset some statistics. This can't be
%    done in the table footer, because the data is dynamic (e.g.,
%    number of glyphs processed) and the table footers are static and
%    do not change based on the table content.
%    \begin{macrocode}
      \@@_handle_table_ending:n {#2}
    \end{longtable}
  \group_end:
}
%    \end{macrocode}
%  \end{macro}


%
%  \begin{macro}{\fonttableglyphcount}
%  \begin{macro}{\g_@@_glyph_int}
%    While generating the font table we count the number of glyphs we
%    see (and typeset). The total is available in the command
%    \cs{fonttableglyphcount} after the table got finished and will be reset to
%    zero when the next table starts.
%    \begin{macrocode}
\DeclareDocumentCommand \fonttableglyphcount {}
                        { \int_use:N \g_@@_glyph_int }
%    \end{macrocode}
%    
%    \begin{macrocode}
\int_new:N \g_@@_glyph_int
%    \end{macrocode}
%  \end{macro}
%  \end{macro}




% \subsubsection{The overall table layout}
%
%     
%  \begin{macro}{\@@_setup_header_footer:nn}
%    Setting up header and footer lines of the table.
%    This macro receives the \textit{font name} and the \textit{font
%    features} specified by the user as its arguments.
%    \begin{macrocode}
\cs_new:Npn \@@_setup_header_footer:nn #1#2{
%    \end{macrocode}
%    On the first page of the table the header may show a caption or
%    some other sort of title based on the value of
%    \cs{l_@@_display_header_bool}. The formatting is handled by
%    \cs{@@_format_table_title:nn} which can be customized through the
%    key \key{title-format}.
%    \begin{macrocode}
    \bool_if:NT \l_@@_display_header_bool
      { \@@_format_table_title:nn{#1}{#2} \@@_debug_nl:n{T}\\*[6pt] }
%    \end{macrocode}
%    We may also want to display a line of hex digits. This is
%    controlled through the key \key{hex-digits} that accepts different
%    values: \kval{head}, \kval{foot}, \kval{head+foo}, \kval{block}
%    (after a block title) or \kval{none}.
%    \begin{macrocode}
    \bool_if:NT \l_@@_header_hex_digits_bool
      { \@@_display_row_of_hex_digits:   \@@_debug_nl:n{H}\\*      }
  \endfirsthead
%    \end{macrocode}
%    Headers for later table pages have a continuation title and
%    maybe a row of hex digits.
%    \begin{macrocode}
    \bool_if:NT \l_@@_display_header_bool
      { \@@_format_table_cont:nn{#1}{#2} \@@_debug_nl:n{T}\\*[6pt] }
    \bool_if:NT \l_@@_header_hex_digits_bool
      { \@@_display_row_of_hex_digits:   \@@_debug_nl:n{H}\\*      }
  \endhead
%    \end{macrocode}
%    Footers of the table are either empty or show a row of hex digits.
%    \begin{macrocode}
    \bool_if:NT \l_@@_footer_hex_digits_bool
      { \@@_display_row_of_hex_digits:   \@@_debug_nl:n{H}\\*      }
  \endfoot
%    \end{macrocode}
%    The footer of the last page of the table will always be
%    empty. Any special row, such as a row of hex digits, will be
%    provided in the table body. The reason is that we may want to
%    display statistics at the very end of the table and those can't be
%    placed into a static footer.
%    \begin{macrocode}
  \endlastfoot
}
%    \end{macrocode}
% \end{macro}
%
%  \begin{macro}{\l_@@_header_hex_digits_bool}
%  \begin{macro}{\l_@@_footer_hex_digits_bool}
%  \begin{macro}{\l_@@_blockwise_hex_digits_bool}
%    Here are the booleans we use in the code.
%    \begin{macrocode}
\bool_new:N \l_@@_header_hex_digits_bool     
\bool_new:N \l_@@_footer_hex_digits_bool
\bool_new:N \l_@@_blockwise_hex_digits_bool
%    \end{macrocode}
%  \end{macro}
%  \end{macro}
%  \end{macro}
%    
%    
%  \begin{macro}{\@@_display_row_of_hex_digits:}
%  \begin{macro}{\@@_format_hex_digit:n}
%    Producing a row of hex digits is simple.
%    \begin{macrocode}
\cs_new:Npn \@@_display_row_of_hex_digits: {
    & \@@_format_hex_digit:n{0}     & \@@_format_hex_digit:n{1} 
    & \@@_format_hex_digit:n{2}     & \@@_format_hex_digit:n{3} 
    & \@@_format_hex_digit:n{4}     & \@@_format_hex_digit:n{5} 
    & \@@_format_hex_digit:n{6}     & \@@_format_hex_digit:n{7} 
    & \@@_format_hex_digit:n{8}     & \@@_format_hex_digit:n{9} 
    & \@@_format_hex_digit:n{A}     & \@@_format_hex_digit:n{B} 
    & \@@_format_hex_digit:n{C}     & \@@_format_hex_digit:n{D} 
    & \@@_format_hex_digit:n{E}     & \@@_format_hex_digit:n{F}  }
%    \end{macrocode}
%    Each digit is typeset in typewriter and in script size. We offer
%    font and color for 
%    customization. Note that it is important to use set an explicit
%    family. Otherwise the hex digits are formatted using the current
%    table font (which may or may not work at all).
%    \begin{macrocode}
\cs_new:Npn \@@_format_hex_digit:n #1 {
  \l_@@_hex_digits_font_tl \l_@@_color_tl #1 }
%    \end{macrocode}
%  \end{macro}
%  \end{macro}
%

%  \begin{macro}{\l_@@_color_tl}
%    The token list to hold definition if set up.
%    \begin{macrocode}
\tl_new:N \l_@@_color_tl
%    \end{macrocode}
%  \end{macro}
%
%    
% \keysetup{overall table}
%    Here are the definitions for the keys used in the code above:
%    \begin{macrocode}
\keys_define:nn {@@} {
%    \end{macrocode}
%    The \key{header} key is a boolean that determines if a header
%    title should be produced (default)
%    \begin{macrocode}
       ,header .bool_set:N   = \l_@@_display_header_bool
       ,header .default:n    = true
       ,header .initial:n    = true
%    \end{macrocode}
%    To ease the setup we also support the key \key{noheader} which is
%    a short form for \texttt{header=false}.
%    \begin{macrocode}
       ,noheader .bool_set_inverse:N = \l_@@_display_header_bool
       ,noheader .default:n          = true
%    \end{macrocode}
%    The default for the \key{title-format} key is to produce a
%    \cs{caption} listing the font name and any features (if
%    given). Note the \cs{IfValueTF} command (provided by
%    \texttt{xparse}) that checks if the second argument got any value
%    or has the special \texttt{--NoValue--} value.
%    \begin{macrocode}
       ,title-format      .cs_set:Np = \@@_format_table_title:nn #1#2
       ,title-format      .initial:n = 
         \IfValueTF{#2} { \caption{ #1~ (features:~ \texttt{\small#2}) } } 
                        { \caption{ #1 } }
%    \end{macrocode}
%    The default continuation title ignores the given features, so the
%    formatting is somewhat simpler. It uses \verb=\caption[]{...}= to
%    make a caption that doesn't alter the table number.
%    \begin{macrocode}
       ,title-cont-format .cs_set:Np = \@@_format_table_cont:nn #1#2
       ,title-cont-format .initial:n = \caption[]{#1~ \emph{cont.}}
%    \end{macrocode}
%    The key \key{hex-digits} is implemented as a choice, where each
%    allowed value sets different booleans that are then used in the code.
%    \begin{macrocode}
       ,hex-digits  .choice:
       ,hex-digits / block   .code:n  =
         \bool_set_true:N   \l_@@_blockwise_hex_digits_bool
         \bool_set_false:N  \l_@@_header_hex_digits_bool
         \bool_set_false:N  \l_@@_footer_hex_digits_bool
       ,hex-digits / foot   .code:n  =
         \bool_set_true:N   \l_@@_footer_hex_digits_bool
         \bool_set_false:N  \l_@@_header_hex_digits_bool
         \bool_set_false:N  \l_@@_blockwise_hex_digits_bool
       ,hex-digits / head   .code:n  =
         \bool_set_true:N   \l_@@_header_hex_digits_bool
         \bool_set_false:N  \l_@@_footer_hex_digits_bool
         \bool_set_false:N  \l_@@_blockwise_hex_digits_bool
       ,hex-digits / head+foot  .code:n  =
         \bool_set_true:N   \l_@@_header_hex_digits_bool
         \bool_set_true:N   \l_@@_footer_hex_digits_bool
         \bool_set_false:N  \l_@@_blockwise_hex_digits_bool
       ,hex-digits / none   .code:n  =
         \bool_set_false:N  \l_@@_header_hex_digits_bool
         \bool_set_false:N  \l_@@_footer_hex_digits_bool
         \bool_set_false:N  \l_@@_blockwise_hex_digits_bool
       ,hex-digits  .initial:n = head
%    \end{macrocode}
%    The font for hex digits are set with \key{hex-digits-font}.
%    \begin{macrocode}
      ,hex-digits-font  .tl_set:N  = \l_@@_hex_digits_font_tl
      ,hex-digits-font  .initial:n = \ttfamily \scriptsize
%    \end{macrocode}
%    The \key{color} key is used in  most places the get colored, some
%    have their own key but default to the main color.         
%    \begin{macrocode}
       ,color .choice:
       ,color / none    .code:n    = \tl_clear:N \l_@@_color_tl
       ,color / unknown .code:n    = \tl_set:Nn \l_@@_color_tl { \color {#1} }
       ,color           .initial:n = blue 
  }
%    \end{macrocode}
%
%
%
%
%  \begin{macro}{\@@_handle_table_ending:n}
%    At the end of the table we may want to display a final row of
%    hex digits and perhaps some statistics, i.e., the number of
%    typeset glyphs.
%    \begin{macrocode}
\cs_new:Npn \@@_handle_table_ending:n #1 {
%    \end{macrocode}
%     
%    \begin{macrocode}
  \@@_debug_nl:n{H} \\*
  \bool_if:NT \l_@@_footer_hex_digits_bool
    { \@@_display_row_of_hex_digits: \@@_debug_nl:n{H} \\*   }
  \bool_if:NT \l_@@_display_statistics_bool
    { \\*[4pt]
      \multicolumn{17}{l}{ \l_@@_stats_font_tl
                           \@@_format_stats:nn{#1}{\fonttableglyphcount} } }
}
%    \end{macrocode}
%
% \keysetup{for statistics}
%    Here are the keys used above. By default we produce statistics.
%    \begin{macrocode}
\keys_define:nn {@@} {
    ,statistics .bool_set:N = \l_@@_display_statistics_bool
    ,statistics .default:n   = true
    ,statistics .initial:n   = true
%    \end{macrocode}
%    the key \key{nostatistics} is just short for \texttt{statistics=false}:
%    \begin{macrocode}
    ,nostatistics .bool_set_inverse:N = \l_@@_display_statistics_bool
    ,nostatistics .default:n = true
%    \end{macrocode}
%   The default font we use is \cs{normalfont}. Again we need to
%    supply a family to avoid getting the font used in the table body. 
%    \begin{macrocode}
    ,statistics-font    .tl_set:N  = \l_@@_stats_font_tl
    ,statistics-font    .initial:n = \normalfont
%    \end{macrocode}
%    And here we have the default text. There is only space for a
%    single line. If more text is needed one needs to provide some
%    explicit \cs{parbox}.
%    \begin{macrocode}
    ,statistics-format  .cs_set:Np = \@@_format_stats:nn #1#2
    ,statistics-format  .initial:n = Total~ number~ of~ glyphs~ in~ #1:~#2 
  }
%    \end{macrocode}
%  \end{macro}
%
%
%
%
%
%  \begin{macro}{\@@_debug_nl:n}
%    While developing the code I had a bit of trouble getting the line
%    endings correct, so I added a little macro that made them visible
%    (displaying its argument in the table margin when the key
%    \key{debug} is used. By default is does nothing.
%    \begin{macrocode}
\cs_new:Npn \@@_debug_nl:n #1 {}
%    \end{macrocode}
%    
% \keysetup{debugging}
%    This key is really internal and is therefore not documented above
%    (and its behavior may changes over time).
%    \begin{macrocode}
\keys_define:nn {@@} {
  debug .code:n = \cs_set:Npn \@@_debug_nl:n ##1
                          {\rlap{\normalfont\scriptsize \qquad ##1}} }
%    \end{macrocode}
%  \end{macro}
%    
%
%
%
%
%
%
% \subsubsection{The producing the table content}
%
%
%    The body of the table consists of rows with sixteen glyphs each
%    and to produce it we loop through all possible Unicode points
%    starting at \texttt{U+0000} and ending with \texttt{U+FFFF}.
%
%    This is implemented a four nested loop that run through the
%    values \texttt{0}, \texttt{1}, \ldots, \texttt{F} with the
%    current hex value in each of the four positions stored in some variable.


%  \begin{macro}{\g_@@_hex_A_tl}
%  \begin{macro}{\g_@@_hex_B_tl}
%  \begin{macro}{\g_@@_hex_C_tl}
%    
%    We really only need three variables the value innermost loop can
%    used directly.
%    \begin{macrocode}
\tl_new:N \g_@@_hex_A_tl
\tl_new:N \g_@@_hex_B_tl
\tl_new:N \g_@@_hex_C_tl
%    \end{macrocode}
%  \end{macro}
%  \end{macro}
%  \end{macro}


%  \begin{macro}{\c_@@_hex_digits_clist}
%    Here is the sequence we loop through on each level.
%    \begin{macrocode}
\clist_const:Nn\c_@@_hex_digits_clist{0,1,2,3,4,5,6,7,8,9,A,B,C,D,E,F}
%    \end{macrocode}
%  \end{macro}



%  \begin{macro}{\@@_produce_table_rows:,
%                \@@_handle_hex_A:n,\@@_handle_hex_B:n,
%                \@@_handle_hex_C:n,\@@_handle_hex_D:n}
%    The overall code layout is then fairly simply:
%    \begin{macrocode}
\cs_new:Npn \@@_produce_table_rows: {
%    \end{macrocode}
%    First to some general initialization
%    \begin{macrocode}
   \@@_initialize_table_rows: 
%    \end{macrocode}
%    and then loop over \cs{\c_@@_hex_digits_clist} for the leftmost
%    hex digit (which we call \enquote{A}).
%    \begin{macrocode}
   \clist_map_function:NN \c_@@_hex_digits_clist \@@_handle_hex_A:n }
%    \end{macrocode}
%
%    Handling \enquote{A} means storing its value for later use and
%    then start a loop for setting the second hex digits:
%    \begin{macrocode}
\cs_new:Npn \@@_handle_hex_A:n #1 { \tl_gset:Nn\g_@@_hex_A_tl{#1}
   \clist_map_function:NN \c_@@_hex_digits_clist \@@_handle_hex_B:n }
%    \end{macrocode}
%
%    Same game for \enquote{B} and \enquote{C}\footnote{Actually this
%    is a white lie. In reality we do a lot of extra stuff when
%    handling \enquote{C} so later one we give a second definition for
%    \cs{@@_handle_hex_C:n} but for understanding the overall picture
%    the simpler one is better.}:
%    \begin{macrocode}
\cs_new:Npn \@@_handle_hex_B:n #1 { \tl_gset:Nn\g_@@_hex_B_tl{#1}
   \clist_map_function:NN \c_@@_hex_digits_clist \@@_handle_hex_C:n }
%    \end{macrocode}
%    
%    \begin{macrocode}
\cs_new:Npn \@@_handle_hex_C:n #1 { \tl_gset:Nn\g_@@_hex_C_tl{#1}
   \clist_map_function:NN \c_@@_hex_digits_clist \@@_handle_hex_D:n }
%    \end{macrocode}
%    In the innermost loop we new have the full Unicode number
%    available, so there we have to decide what to do with it. This is
%    done by \cs{@@_handle_hex_D:n} that receives the full number,
%    e.g., \texttt{1A7C} as its argument.
%    \begin{macrocode}
\cs_new:Npn \@@_handle_hex_D:n #1 {
  \@@_handle_slot:x
     { " \g_@@_hex_A_tl \g_@@_hex_B_tl \g_@@_hex_C_tl #1 }
}
%    \end{macrocode}
%  \end{macro}



%  \begin{macro}{\@@_handle_slot:n,\@@_handle_slot:x}
%    If the current slot number under inspection contains a glyph in
%    our font we want to typeset it. But we don't do this immediately,
%    instead we build up the whole row and typeset it later.  We
%    therefore append a \verb=&= and the glyph (including the necessary
%    formatting) to the token list \cs{g_@@_row_tl}.
%    \begin{macrocode}
\cs_new:Npn \@@_handle_slot:n #1 {
   \@@_if_uchar_exists:nTF { #1 }
      { \tl_gput_right:Nn \g_@@_row_tl
                          { & \@@_format_glyph:n { \symbol{#1} } }
%    \end{macrocode}
%    We then increment the overall glyph count and record that we have
%    seen at least one glyph in the current row. There is no much
%    point in displaying rows that are completely empty because
%    otherwise we end up with extremely large tables which are
%    basically empty.
%    \begin{macrocode}
        \int_gincr:N\g_@@_glyph_int
        \bool_gset_true:N \g_@@_glyph_seen_bool
      }
%    \end{macrocode}
%    If the current slot has no glyph in the font we also add a
%    \verb=&= followed by an something that indicates the glyph is
%    missing.  If we do font comparison, it may show the glyph from
%    the other font (if it exists there) in some special way to indicate which glyph
%    should be in this slot.
%    \begin{macrocode}
      { \@@_handle_missing_glyph:n {#1} }
}
%    \end{macrocode}
%    
%    \begin{macrocode}
\cs_generate_variant:Nn \@@_handle_slot:n {x}
%    \end{macrocode}
%  \end{macro}
%

%  \begin{macro}{\@@_handle_missing_glyph:n,
%                \@@_handle_missing_glyph_std:n,
%                \@@_handle_missing_glyph_compare:n}
%    
%    In the standard case we typeset a special symbol to indicate that the glyph is missing.
%    For this case we provide some customization through keys:
%    \cs{l_@@_missing_glyph_tl} holds the symbol for a missing glyph
%    (default a hyphen). It is typeset in a specific color and we allow for
%    setting it in a special font. The actual symbol number in
%    \verb=#1=is not needed in this scenario.
%    \begin{macrocode}
\cs_new:Npn \@@_handle_missing_glyph_std:n #1 {
 \tl_gput_right:Nn \g_@@_row_tl 
    { &
      \@@_format_glyph:n {
%        \colorbox{black!30}                      % <--- povide interface
                 {\l_@@_missing_glyph_color_tl
                           \l_@@_missing_glyph_font_tl 
	                   \l_@@_missing_glyph_tl }
      }
    }
}
%    \end{macrocode}
%
% \keysetup{missing glyphs}
%    Here are the keys for customizing the missing glyph representation.
%    \begin{macrocode}
\keys_define:nn {@@} {
   missing-glyph-color .choice:
  ,missing-glyph-color / none    .code:n    =
       \tl_clear:N \l_@@_missing_glyph_color_tl
  ,missing-glyph-color / unknown .code:n    =
       \tl_set:Nn \l_@@_missing_glyph_color_tl { \color {#1} }
%
  ,missing-glyph-font  .tl_set:N  = \l_@@_missing_glyph_font_tl
  ,missing-glyph-font  .initial:n = \ttfamily \scriptsize
  ,missing-glyph       .tl_set:N  = \l_@@_missing_glyph_tl
  ,missing-glyph       .initial:n = -  }
%    \end{macrocode}
%
%
%    The default definition for the color is to use the same as the
%    one specified by the \key{color} key. We therefore define the
%    default outside of \pkg{l3keys} method.
%    \begin{macrocode}
\tl_new:N  \l_@@_missing_glyph_color_tl
\tl_set:Nn \l_@@_missing_glyph_color_tl {\l_@@_color_tl}
%    \end{macrocode}
%
%    This here is the version that handles a missing glyph by checking
%    the \key{compare-with} font to see if that font contains the
%    glyph.
%    If yes, the substitute glyph will be typeset, otherwise the missing
%    glyph symbol is shown by calling \cs{@@_handle_missing_glyph_std:n}.
%    \begin{macrocode}
\cs_new:Npn \@@_handle_missing_glyph_compare:n #1 {
  \group_begin:
%    \end{macrocode}
%    Locally switch  to the other font, then check for the glyph:
%    \begin{macrocode}
    \l_@@_compare_font_tl
    \@@_if_uchar_exists:nTF { #1 }
       {
%    \end{macrocode}
%    If available format it (together with the \texttt{\&}) but use a
%    special color and perhaps a background color.
%    \begin{macrocode}
         \tl_gput_right:Nn \g_@@_row_tl 
            { &
              \@@_format_glyph:n
                 { \l_@@_compare_bgcolor_tl { \l_@@_compare_color_tl
                                              \l_@@_compare_font_tl
                                              \symbol {#1} } 
                 }
	    }
%    \end{macrocode}
%    Also tell the algorithm that we have seen a glyph to typeset. If
%    we don't do this then a row consitingon only of substitute glyphs is not
%    typeset. However, we don't update the glyph count, because this
%    is not a glyph from the main font we display.
%    \begin{macrocode}
         \bool_gset_true:N \g_@@_glyph_seen_bool
       }
%    \end{macrocode}
%    If the alternate font doesn't have the glyph either we
%    typesetting the missing glyph symbol.
%    \begin{macrocode}
       { \@@_handle_missing_glyph_std:n {} }
  \group_end:
}
%    \end{macrocode}
%    
% \keysetup{comparison}

%    \begin{macrocode}
\tl_new:N \l_@@_compare_color_tl
\tl_new:N \l_@@_compare_bgcolor_tl
%    \end{macrocode}
%    
%    \begin{macrocode}
\keys_define:nn {@@}
     {
       ,compare-with .tl_set:N  = \l_@@_compare_with_tl
       ,compare-with .initial:n = 
       ,compare-color .choice:
       ,compare-color / none    .code:n    = \tl_clear:N \l_@@_compare_color_tl
       ,compare-color / unknown .code:n    = \tl_set:Nn  \l_@@_compare_color_tl { \color {#1} }
       ,compare-color           .initial:n = red
       ,compare-bgcolor .choice:
       ,compare-bgcolor / none    .code:n    = \tl_clear:N \l_@@_compare_bgcolor_tl
       ,compare-bgcolor / unknown .code:n    = \tl_set:Nn  \l_@@_compare_bgcolor_tl { \colorbox {#1} }
       ,compare-bgcolor           .initial:n = black!10
     }
%    \end{macrocode}
%
%
%    By default, i.e., if no font for comparison has been specified we
%    handle missing glyphs by displaying a missing glyph symbol.
%    \begin{macrocode}
\cs_new_eq:NN \@@_handle_missing_glyph:n
              \@@_handle_missing_glyph_std:n
%    \end{macrocode}
%  \end{macro}
%


%  \begin{macro}{\@@_format_glyph:n}
%    Every glyph is typeset in a box of equal width with the glyph
%    centered and if necessary protruding on both sides.
%    \begin{macrocode}
\cs_new:Npn \@@_format_glyph:n #1 {
  \hbox_to_wd:nn {\l_@@_glyph_box_dim} { \hss #1 \hss } }
%    \end{macrocode}
%    
% \keysetup{glyph typesetting}
%    The key to customize the width. The 6pt are fine for most cases.
%    \begin{macrocode}
\keys_define:nn {@@} {
  glyph-width .dim_set:N = \l_@@_glyph_box_dim
 ,glyph-width .initial:n = 6pt    }
  
%    \end{macrocode}
%  \end{macro}


%  \begin{macro}{\@@_if_uchar_exists:n}
%    For testing whether or not a slot position contains a glyph we
%    need to resort to low-level methods, because so far the
%    \texttt{expl3} doesn't offer an interface.
%    \begin{macrocode}
\prg_set_conditional:Npnn \@@_if_uchar_exists:n #1 { TF }
  { \tex_iffontchar:D \tex_font:D #1 \scan_stop:
      \prg_return_true:
    \else:
      \prg_return_false:
    \fi:
  }
%    \end{macrocode}
%  \end{macro}



%  \begin{macro}{\@@_handle_hex_C:n}
%    
%    \begin{macrocode}
\cs_set:Npn \@@_handle_hex_C:n #1 {
  \@@_maybe_display_block_title:
%
  \tl_gset:Nn\g_@@_hex_C_tl{#1}
  %
  \@@_update_block_title:n { \g_@@_hex_A_tl
                             \g_@@_hex_B_tl
                             \g_@@_hex_C_tl }
  \int_compare:nNnF
      { " \g_@@_hex_A_tl \g_@@_hex_B_tl \g_@@_hex_C_tl 0 }
    < { "\l_@@_range_start_tl }
      {
        \tl_gset:Nx \g_@@_row_tl
           {
             \exp_not:N \@@_format_row_title:n
                 { \g_@@_hex_A_tl \g_@@_hex_B_tl \g_@@_hex_C_tl  }
           }
        \int_compare:nNnF
           { " \g_@@_hex_A_tl \g_@@_hex_B_tl \g_@@_hex_C_tl 0 }
         > { "\l_@@_range_end_tl }
           { \clist_map_function:NN \c_@@_hex_digits_clist
                                    \@@_handle_hex_D:n     }
      }
}
%    \end{macrocode}
%  \end{macro}

   %    \begin{macrocode}






\bool_new:N \l_@@_display_block_bool
\tl_new:N   \l_@@_display_block_action_tl


% \keysetup{display blocks}
   
\keys_define:nn {@@}
     {
       ,display-block .choice:
       ,display-block / titles  .code:n =
         \bool_set_true:N \l_@@_display_block_bool
         \tl_set:Nn \l_@@_display_block_action_tl {\\}
       ,display-block / rules   .code:n =
         \bool_set_false:N \l_@@_display_block_bool
         \tl_set:Nn \l_@@_display_block_action_tl {\\ \midrule}
       ,display-block / none   .code:n =
         \bool_set_false:N \l_@@_display_block_bool
         \tl_set:Nn \l_@@_display_block_action_tl {\\}
       ,display-block .initial:n  = titles
  }
%    \end{macrocode}



     
%
%

%
% \keysetup{ranges}
%    \begin{macrocode}
\keys_define:nn {@@}
     {
       ,range-start .tl_set:N  = \l_@@_range_start_tl
       ,range-start .initial:n = 0000
       ,range-end   .tl_set:N  = \l_@@_range_end_tl
       ,range-end   .initial:n = FFFF
     }


     

\tl_new:N \g_@@_row_tl
\tl_new:N \g_@@_block_title_tl
%    \end{macrocode}



%  \begin{macro}{\g_@@_first_row_bool,\g_@@_glyph_seen_bool,\g_@@_row_missing_bool}
%    Declare the three booleans used in the code below. The will tell
%    us answers to the following questions
%    \begin{itemize}
%    \item Are we processing the first row?
%    \item Have we seen any glyph so far?
%    \item Did we have one or more missing rows recently?
%    \end{itemize}
%    \begin{macrocode}
\bool_new:N \g_@@_first_row_bool
\bool_new:N \g_@@_glyph_seen_bool
\bool_new:N \g_@@_row_missing_bool
%    \end{macrocode}
%  \end{macro}






%  \begin{macro}{\@@_initialize_table_rows:}
%    At the start of a table we are processing the first row
%    and so we (obviously) haven't see a glyph yet and there wasn't a
%    missing row recently.
%    \begin{macrocode}
\cs_new:Npn \@@_initialize_table_rows: {
     \bool_gset_true:N \g_@@_first_row_bool
     \bool_gset_false:N \g_@@_glyph_seen_bool     
     \bool_gset_false:N \g_@@_row_missing_bool
%    \end{macrocode}
%    And clearly the glyph count for the font is zero.
%    \begin{macrocode}
     \int_gzero:N \g_@@_glyph_int 
}
%    \end{macrocode}
%  \end{macro}


  
%    
%    \begin{macrocode}

\errorstopmode
\errorcontextlines=50

\cs_new:Npn \@@_maybe_display_block_title: {
  \bool_if:NTF \g_@@_glyph_seen_bool
    {
      \tl_if_empty:NTF \g_@@_block_title_tl
         {
           \bool_if:NTF \g_@@_row_missing_bool
                { \@@_debug_nl:n{A}\\[6pt] }
                { \@@_debug_nl:n{B}\\      }
         }
         {
           \typeout{ Processing~ \tl_use:N \g_@@_block_title_tl }
           \bool_if:NTF \l_@@_display_block_bool
                {
                  \bool_if:NTF \g_@@_first_row_bool
                    {
                      \bool_gset_false:N \g_@@_first_row_bool
                      \@@_debug_nl:n{C}\\[-8pt]
                    }
                    {
                      \@@_debug_nl:n{D}\\[4pt]
                    }
	          \multicolumn{17}{c}{\normalfont \bfseries \tl_use:N \g_@@_block_title_tl}
                  \bool_if:NTF \l_@@_blockwise_hex_digits_bool
	               { \@@_debug_nl:n{E}\\*
                         \@@_display_row_of_hex_digits:
                         \@@_debug_nl:n{H}\\*
                       }
                       { \@@_debug_nl:n{F}\\*[2pt] }
                }
                {
                  \bool_if:NTF \g_@@_first_row_bool
                    { \bool_gset_false:N \g_@@_first_row_bool }
                    {
                      \@@_debug_nl:n{G~ (new~ block)}
                      \l_@@_display_block_action_tl
                    }
                }
	   \tl_gclear:N \g_@@_block_title_tl
	 }
      \bool_gset_false:N \g_@@_glyph_seen_bool     
      \bool_gset_false:N \g_@@_row_missing_bool
      \tl_use:N \g_@@_row_tl
    }
%    \end{macrocode}
%    Current row had no chars, remember that fact
%    \begin{macrocode}
    {
      \bool_gset_true:N \g_@@_row_missing_bool
    }
}





\cs_new:Npn \@@_update_block_title:n #1 {
    \tl_gset:Nx \g_@@_block_title_tl  {
    \int_case:nnF{ "#1 }
       {
         { "000 }{ Basic~ Latin }
         { "008 }{ Latin-1~ Supplement }
         { "010 }{ Latin~ Extended-A }
         { "018 }{ Latin~ Extended-B }
         { "025 }{ IPA~ Extension }
         { "02B }{ Spacing~ Modifier~ Letters }
         { "030 }{ Combining~ Diacritical~ Marks }
         { "037 }{ Greek~ and~ Coptic }
         { "040 }{ Cyrillic }
         { "053 }{ Armenian }
         { "059 }{ Hebrew }
         { "060 }{ Arabic }
         { "070 }{ Syriac }
         { "075 }{ Arabic~ Supplement }
         { "078 }{ Thaana }
         { "07C }{ NKo }
         { "090 }{ Devanagari }
         { "098 }{ Bengali }
         { "0A0 }{ Gurmukhi }
         { "0A8 }{ Gujarati }
         { "0B0 }{ Oriya }
         { "0B8 }{ Tamil }
         { "0C0 }{ Telugu }
         { "0C8 }{ Kannada }
         { "0D0 }{ Malayalm }
         { "0D8 }{ Sinhala }
         { "0E0 }{ Thai }
         { "0E8 }{ Lao }
         { "0F0 }{ Tibetan }
         { "100 }{ Myanmar }
         { "10A }{ Georgian }
         { "110 }{ Hangul~ Jamo }
         { "120 }{ Ethiopic  }
         { "138 }{ Ethiopic~ Supplement }
         { "13A }{ Cherokee }
         { "140 }{ Unified~ Canadian~ Aboriginal~ Syllabics }
         { "168 }{ Ogham }
         { "16A }{ Runic }
         { "170 }{ Tagalog }
         { "172 }{ Hanunoo }
         { "174 }{ Buhid }
         { "176 }{ Tagbanwa }
         { "178 }{ Khmer }
         { "180 }{ Mongolian }
         { "190 }{ Limbu }
         { "195 }{ Tai~ Le }
         { "198 }{ New~ Tai~ Le }
         { "19E }{ Khmer~ Symbols }
         { "1A0 }{ Buginese }
         { "1B0 }{ Balinese }
         { "1D0 }{ Phonetic~ Extensions }
         { "1D8 }{ Phonetic~ Extensions~ Supplement }
         { "1DC }{ Combining~ Diacritical~ Marks~ Supplement }
         { "1E0 }{ Latin~ Extended~ Additional }
         { "1F0 }{ Greek~ Extended }
         { "200 }{ General~ Punctuation }
         { "207 }{ Superscripts~ and~ Subscripts }
         { "20A }{ Currency~ Symbols }
         { "20D }{ Combining~ Diacritical~ Marks~ for~ Symbols }
         { "210 }{ Letterlike~ Symbols }
         { "215 }{ Number~ Forms }
         { "219 }{ Arrows }
         { "220 }{ Mathematical~ Operators }
         { "230 }{ Miscellaneous~ Technical }
         { "240 }{ Control~ Pictures }
         { "244 }{ Optical~ Character~ Recognition }
         { "246 }{ Enclosed~ Alphanumerics }
         { "250 }{ Box~ Drawing }
         { "258 }{ Block~ Elements }
         { "25A }{ Geometric~ Shapes }
         { "260 }{ Miscellaneous~ Shapes }
         { "270 }{ Dingbats }
         { "27C }{ Miscellaneous~ Mathematical~ Symbols-A }
         { "280 }{ Braille~ Patterns }
         { "290 }{ Supplement~ Arrows-B }
         { "298 }{ Miscellaneous~ Mathematical~ Symbols-B }
         { "2A0 }{ Supplement~ Mathematical~ Operators }
         { "2B0 }{ Miscellaneous~ Symbols~ and~ Arrows }
         { "2C0 }{ Glagolitic }
         { "2C6 }{ Latin~ Extended-C }
         { "2C8 }{ Coptic }
         { "2D0 }{ Georgian~ Supplement }
         { "2D3 }{ Tifinagh }
         { "2D8 }{ Ethiopic~ Extended }
         { "2E8 }{ CJK~ Radicals~ Supplement }
         { "2F0 }{ Kangxi~ Radicals }
         { "2FF }{ Ideographic~ Description~ Characters }
         { "300 }{ CJK~ Symbols~ and~ Punctuation }
         { "304 }{ Hiragana }
         { "30A }{ Katakana }
         { "310 }{ Bopomofo }
         { "313 }{ Hangul~ Compatibility~ Jamo }
         { "319 }{ Kanbun }
         { "31A }{ Bopomofo~ Extended }
         { "31C }{ CJK~ Strokes }
         { "31F }{ Katakana~ Phonetic~ Extensions }
         { "320 }{ Enclosed~ CJK~ Letters~ and~ Months }
         { "330 }{ CJK~ Compatibility }
         { "4DC }{ Yijing~ Hexagram~ Symbols }
         { "A00 }{ Yi~ Syllables }
         { "A49 }{ Yi~ Radicals }
         { "A70 }{ Modifier~ Tone~ Letters }
         { "A72 }{ Latin~ Extended-D }
         { "A80 }{ Syloti~ Nagri }
         { "A84 }{ Phags-pa }
         { "E00 }{ Private~ Use~ Area }
         { "F90 }{ CJK~ Compatibility~ Ideographs }
         { "FB0 }{ Alphabetic~ Presentation~ Forms }
         { "FB5 }{ Arabic~ Presentation~ Forms-A }
         { "FE0 }{ Variation~ Selectors }
         { "FE1 }{ Vertical~ Forms }
         { "FE2 }{ Combining~ Half~ Marks  }
         { "FE3 }{ CJK~ Compatibility~ Forms }
         { "FE5 }{ Small~ Form~ Variants }
         { "FE7 }{ Arabic~ Presentation~ Forms-B }
         { "FF0 }{ Halfwidth~ and Fullwidth~ Forms }
         { "FFF }{ Specials~ ... }
%...         
         { "1000 }{ Plane 1~ ... }
       }
%    \end{macrocode}
%    If none of the above has matched we are somewhere within a block
%    so we want keep the current name. However, since the case statement was
%    executed within a |\tl_gset:Nx| we have to do this by passing the current block name
%    back.
%    \begin{macrocode}
       { \tl_use:N \g_@@_block_title_tl }
  }
}





%\cs_new:Npn \@@_if_ignore_slot:nTF #1#2#3 {\ifodd"#1\relax #2\else #3\fi}







%\unicodefonttabletablesetup{debug}


\cs_new:Npn \@@_format_row_title:n #1 {
  \texttt { \footnotesize \l_@@_color_tl U+#1 0 \, - \, #1 F }
}


\ExplSyntaxOff




\RequirePackage{longtable,booktabs,caption,fontspec}




%    \end{macrocode}
%
%
%  So what remains to be done is executing all options passed to the
%    package via \cs{usepackage}.
%    \begin{macrocode}
\ProcessKeysPackageOptions{@@}
%</package>
%    \end{macrocode}
%
%
%
%
%
% \subsection{The samples file}
%
%    \begin{macrocode}
%<*samples>
%    \end{macrocode}
%
%    \begin{macrocode}
%<<VERBATIMLINE
%!TEX program = lualatex

%VERBATIMLINE
%    \end{macrocode}
%
%    \begin{macrocode}
\documentclass{article}

\usepackage{xparse,color}

\usepackage{fontspec}

\setmainfont{Linux Biolinum O}
\setmonofont{SourceCodePro}

\usepackage{unicodefonttable}

\addtolength\textwidth{30pt}

\begin{document}

\listoftables


\section{Computer Modern  --- 8bit font}

\displayfonttable*[color=none,
  range-end = 7F,
]{cmr10}

%\section{Computer Modern Sans --- 8bit font} \displayfonttable*[]{cmss10}

\newpage

\section{TeX Gyre Heros (Helvetica)  --- 8bit font}

\displayfonttable*[color=red,nostatistics=false,
  hex-digits = head+foot,
  range-end = FF,
]{ec-qhvr}

\newpage

\section{Latin Modern Sans --- OTF font}

\displayfonttable[
%  display-block = rules,
  %              missing-glyph = \tiny\setlength\fboxsep{0pt}\fbox{$\times$},
  hex-digits = block,
  title-cont-format = \caption[]{\emph{continued}},
]{Latin Modern Sans}

\newpage

\section{\TeX{} Gyre Pagella (Palatino) oldstyle figures --- OTF font}

\displayfonttable{TeX Gyre Pagella}[Numbers=OldStyle]

\end{document}
%    \end{macrocode}
%
%    \begin{macrocode}
%</samples>
%    \end{macrocode}



%
%
% \subsection{The standalone file}
%
%    \begin{macrocode}
%<*standalone>
%    \end{macrocode}
%
%    \begin{macrocode}
\typeout{CODE here}
%    \end{macrocode}
%
%
%    \begin{macrocode}
%</standalone>
%    \end{macrocode}
%
% \Finale
%


\endinput

%%%%%%%%%%%%%%%%%%%

%    \begin{macrocode}
\cs_new_protected:cpn { \c__keys_props_root_str .cs_set:Np } #1
   { \__keys_cs_make:NNpn \cs_set:Npn #1 { } }
\cs_new_protected:Npn \__keys_cs_make:NNpn #1#2#3#
   {
     \cs_set_protected:cpx { \c__keys_code_root_str \l_keys_path_str } ##1
       { #1 \exp_not:N #2 \exp_not:n {#3} {##1} }
     \use_none:n
   }  
%    \end{macrocode}
%    

%%%%%%%%%%%%%%%%%%%

