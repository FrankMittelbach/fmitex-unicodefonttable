\documentclass{ltugboat}

\usepackage{fontspec}
\setmainfont{Alegreya}
\setsansfont{Alegreya Sans}


\usepackage{microtype,graphicx,amssymb,csquotes,unicodefonttable}

\newcommand\LPack[1]{\texttt{#1}}
\newcommand\Prog[1]{\texttt{#1}}
\newcommand\pdfTeX{pdf\TeX}

\newcommand\tofu{$\square$}

\providecommand\pkg[1]{\texttt{#1}}



\setcounter{secnumdepth}{1}


\author{Frank Mittelbach}
\title{Preventing Tofu with \pdfTeX{} and Unicode engines}
\address{Mainz, Germany}
\netaddress{https://www.latex-project.org}
\personalURL{https://ctan.org/pkg/unicodefonttable}

\begin{document}

\maketitle

\begin{abstract}

\end{abstract}

\tableofcontents

\section{With tofu through the years}

Tofu is not just an essential ingredient for many Asian dishes, it is
also the nick name for little squares that are produced by many
bowsers when they get asked to render a character for which they do
not have a glyph available.

Especially in the early days of the World Wide Web,
websites in foreign languages (from the perspective of your computer)
got often littered with such squares, making text comprehension quite
difficult if not impossible in some cases. So instead of getting
\begin{quote}
  ?`But aren't Kafka's Schlo{\ss} \&
  {\AE}sop's {\OE}uvres often na{\"\i}ve vis-\`{a}-vis the d{\ae}monic
  \textit{ph{\oe}nix's official r\^{o}le} in fluffy souffl\'{e}s?
\end{quote}
you might have seen something like
\begin{quote}
  \tofu But aren't Kafka's Schlo{\tofu} \tofu{} {\tofu}sop's
  \tofu{}uvres often na{\tofu}ve vis-\tofu{}-vis the d{\tofu}monic
  \textit{ph{\tofu}nix's official r\tofu{}le} in fluffy
  souffl\tofu{}s?
\end{quote}
Over the years the situation with browsers improved (partly because
using inferior fonts deemed acceptable as long as they could render
the needed glyphs), but even nowadays you may find tofu-littered sites
when you direct your browser to it or perhaps worse those where you
browser thinks it can show you the glyphs but renders the wrong ones.

While with browsers you may accept a certain imperfection in the
rendering, tofu in printed material is quite unacceptable and
typesetting systems should always use the correct glyphs or at least
tell you very explicitly if they are unable to do so for some reason,
to allow you to apply some corrective actions. In the remainder of
this article we will discuss how \TeX{} and in particular \LaTeX{} is
doing in this respect and what a user can or has to do to avoid some
capital blunder.

%\subsection{The first \TeX{} years}
\subsection{Early  vegetarian dishes with \TeX{}}

In the early days of \TeX{} the use of fonts was easy because you
could use any font you wanted as long as it was called Computer
Modern.

In other words there was essentially only one set of fonts available
for use with \TeX\ and the glyphs it contained and how to address them
was described in the \TeX{}book~\cite{Knuth:ct-a}. Furthermore, all fonts only
contained 128 glyphs, i.e., essentially the base Latin characters, a
few accents to construct diacritical characters using the \cs{accent}
primitive and a few other symbols such as \textdagger, \$ and so forth
to be accessed through command names.

Thus, once you learned the construction methods and memorized the
control sequences for accessing the existing symbols you could be sure
that the characters you used would faithfully appear in the printed
result. Of course, part of the reason was that the glyph set was
fairly limited and already for any Latin-based language other than
English posed series issues, be it that necessary glyphs where missing
or only available as constructed characters (whenever accents where
involved) which prevented \TeX{} from applying hyphenation.

So when \TeX{} got more and more popular outside the English-speaking
world there was considerable pressure on Don Knuth (largely by
European users, the author among them) to extend \TeX{} so that it
could better handle languages with larger character sets. At the 1989
\TeX{} conference in Stanford we finally managed to convince Don to
reopen (in a limited way) \TeX{} development and to produce
\TeX~3. This version of \TeX{} was then able to deal with more than
one language within a document (e.g., use multiple hyphenation
patterns) and support 8-bit input and output (that is 256 characters
in a font).

While this enabled the use of different input code pages, as common in
those days in computers depending on the country and solved the
problem of hyphenating words containing accented characters (by using
fonts with precomposed glyphs), it also posed new challenges.

Depending on the code page used in the computer when writing a
document the same keyboard character might got associated with a
different number (between 0 and 255) and that number had to be mapped
to the right slot in a font to produce the glyph that was originally
intended. So the days of input number equals font glyph position was
definitely over and the \TeX{} world had to come up with a more
elaborate scheme to translate one into the other to avoid missing or
wrong characters in the output.


\subsection{The \LaTeXe{} solution}

For \LaTeX{} the solution came in form of the New Font Selection
Scheme~\cite{Mittelbach:TB11-1-91} and in particular with the packages
\LPack{inputenc} (for managing input in different code pages and
mapping it to a standard internal
representation) and \LPack{fontenc}
(for translating this internal representation to the correct glyph
positions in different fonts).


\subsubsection{Introducing font encodings}

\LaTeX{} classified the font encodings and gave them names such as
\texttt{OT1}, \texttt{T1}, \texttt{TS1}, \texttt{T2A}, \texttt{T2B},
etc. Each such font encoding defined which glyphs are in a font using
that encoding and where each glyph was stored in the font. Thus, if
you had two different fonts with the same encoding you could exchange
one for the other and still be one hundred percent sure\footnote{Well,
  more 99\% because sometimes fonts claimed to be in one encoding but
  didn't faithfully follow its specification, e.g., didn't provide all
  glyphs or sometimes even placed wrong glyphs into some slots.}  that
your document  typesets correctly without any missing or
incorrect glyphs in the output.

In practice only a small number of font encodings ever got used and
new fonts usually were made available in these ``popular'' encodings
by providing the necessary font re-encodings through the virtual font
mechanism or through re-encodings done by device drivers such as
\Prog{dvips} or directly in the engine (in case of \Prog{pdftex}).

As an overall result, life for \LaTeX{} users got again fairly easy after
1994 and remained this way well into this century, because just by
specifying which font encoding to use documents normally would typeset
without any defects regardless of the font family that got used. And
due to the fact that for users writing in Latin-based languages
essentially every font available was provided in \texttt{T1} encoding,
it was also clear which glyphs were available and those were then
available universally.

\subsubsection{Pitfalls with missing input encodings}

Of course, there was also the need to specify the input encoding---at
least if one wanted to input accented characters directly from the
keyboard instead of using constructs like \verb=\"a=. One problem in
this respect was that, depending on the language you were writing in,
it sometimes worked even without specifying the input encoding. This
was possible because the \texttt{T1} font encoding was nearly identical to
the quite common \texttt{latin1} input encoding.\footnote{For example,
  with French texts it worked throughout. However, with German only
  the ``umlauts'' worked, but the sharp s ``ß'' generated a different
  character.}  Years later, omitting the input encoding declaration
even when it worked initially finally backfired: once \LaTeX{} moved
on to make UTF-8 the default encoding, documents stored in legacy
encodings failed if they didn't contain an input
declaration.\footnote{The remedy for such old documents is then to
  either add the missing declaration or re-encode the old source and
  store it in UTF-8.}

\subsubsection{Pitfalls with the \texttt{TS1} encoding}

When 8-bit fonts became more common, the \TeX{} community defined two
font encodings during a conference at Cork in 1990. These are
\texttt{T1}, which holds common Latin text glyphs that play a role in
hyphenation and therefore have to be present in the same font when
seen by \TeX{}, and the \texttt{TS1} encoding, which contains other
symbols, such as oldstyle numerals or currency symbols, that can be
fetched from as secondary font without harm to the hyphenation
algorithm (because they do not appear as part of words to be
hyphenated).

The glyphs in the \texttt{T1} encoding were well chosen and it is usually
possible to arrange any commercial or free font to be presented in
this encoding to \TeX.\footnote{There are a few exceptions where a some
  glyphs are missing, but these cases are rare.} As a result replacing
fonts encoded in \texttt{T1} means that you can be fairly sure that
there will be no tofu in your output afterwards.

Unfortunately, this is not at all true for the \texttt{TS1}
encoding. Here the community made a big mistake by going overboard in
adding several \enquote{supposedly} useful glyphs to the encoding that could
be produced in theory (and for Computer Modern and similar \TeX{}
fonts were produced), but that simply did not exist in any font that
had its origin outside the \TeX{} world.

As a result, using such glyphs from the \texttt{TS1} encoding meant
that you either had to stay with a very limited number of font
families or you had to be very careful not to use any of the
problematical symbols to avoid tofu.

To ease this problematical situation, the \texttt{TS1} encoding was
subdivided into five sub-encodings and an interface was established to
identify that a font family with a certain NFSS name belonged one of
the sub-encodings. This way \LaTeX{} was enabled to make
\enquote{reasonable} adjustments when a requested symbol was not
available in the current font, either by substituting it from a
different font or by telling you, that the symbol is not there by
means of an error message\Dash not perfect but better than tofu in the
end. This was implemented in the \pkg{textcomp} package that provided
the \LaTeX{} commands to access the symbols from \texttt{TS1}.

In one of the recent \LaTeX{} releases the code from \pkg{textcomp}
was moved to the \LaTeX{} format, so that these extra symbols are now
available out of the box without the need to load an additional
package. At the same time the classification of fonts into
\texttt{TS1} sub-encodings was reworked. We now support nine
sub-encodings and the \LaTeX{} format contains close to 200
declarations that sort the commonly available font families into the
right sub-encodings. Thus these days the situation is fairly well under
control again (at least with \pdfTeX).

\section{Unicode}

One of the goals of Unicode is to uniquely identify each and every
character used in different languages and scripts around the world,
thereby avoiding some of the possible translation problems that
occurred because a text was written under the assumption of one
(8-bit) encoding, but interpreted later on under a different encoding.

While this was a huge step forward for correctly interpreting any
source document (because it took away with different input
encodings\Dash all is now Unicode), it unfortunately reintroduced a lot
of tofu through the back door.

The reason is simple: with Unicode as the means to reliably address a
glyph to be typeset in a font, such a font has to contain glyphs for
\emph{all} characters available in Unicode, because \TeX{} just takes
the Unicode number and tells the current font \emph{typeset this}.
While this is in theory possible in the TrueType or OpenType font
formats there is no single font that offers this.\footnote{The only
  font I know that comes close is Code\,2000~\cite{web:Code2000} that implements 53068
  characters\Dash but even that is only a fraction of the whole
  Unicode universe.  Google's Noto project~\cite{web:Noto}, which stands for
  \enquote{\textbf{no to}fu} and which was established to develop fonts
  for typesetting text in any of the worlds languages and scripts is
  currently split across more than a thousand font families, e.g., if
  you want to typeset in Latin you can use \textsf{Noto Sans}, but for
  Japanese you need \textsf{Noto Sans Japanese} and so forth.}
\LaTeX{} has no way to identify if glyphs are missing, because the
typesetting of paragraph text is a very low-level process in \TeX{} and
in contrast to the \pdfTeX{} engine where \LaTeX{} can reliably
assume that a font in \texttt{T1} encoding implements the whole
encoding, in Unicode engines all fonts are in the \texttt{TU} encoding
(the whole of Unicode), but no font implements that without huge holes.

In theory it would have been possible to devise sub-encodings of
\texttt{TU} and assign each and every font to the appropriate
sub-encoding, but in practice this would be a hopeless undertaking,
because each and every font implements its own set of glyphs so that
no useful classification is possible.

Thus when you typeset in \XeTeX{} or \LuaTeX{} and you request using a
certain font family with something like
\begin{verbatim}
  \setmainfont{Alegreya}
\end{verbatim}
you have to hope your chosen family contains glyphs for all characters
that you intend to use in your document: if not, you will end up with
tofu in one or the other place.

To give you some figures: \textsf{Latin Modern Roman} (the default
font in \LaTeX{} on Unicode engines) implements 794 characters, the
\textsf{Optima} font on the Mac just 264, \textsf{Alegreya} (used for
this document) 1251 and \textsf{Noto Serif} even 2840. But none of this
means that the characters contained in your document are covered.


\subsection{Letting \TeX{} tell you about your tofu}

The \TeX{} program offers one tracing parameter, called
\cs{tracinglostchars}, that, if set to a positive value, reports missing
glyphs (a.k.a.\ tofu) in the transcript file with a line such as
\begin{verbatim}
Missing character: There is no
             È (U+00C8) in font cmr10!
\end{verbatim}
Interestingly enough, this information is not even given by default, but
only when you explicitly ask for it\Dash obviously, Don Knuth did not
foresee that \TeX{} is used with fonts other than those carefully
crafted for \TeX{} and containing all the characters you may want.

Recently all \TeX{} engines got enhanced so that tofu reporting
becomes a little better: you can now set this parameter to \texttt{2}
after which it reports its finding also on the terminal (the new
default value in \LaTeX{}), or you can set it to \texttt{3} after
which it will throw an error rather than a warning that is easy to
miss.
%
With Unicode engines we strongly recommend to always set
\begin{verbatim}
  \tracinglostchars=3
\end{verbatim}
in the preamble of your document\Dash it is much better to get errors
when writing your documents instead of getting reports by others about
tofu in your published work.  As explained before, when typesetting with
\pdfTeX{} there is little danger of ending up with tofu, so there it is
less important to change the parameter, though it obviously doesn't
hurt.


\section{Typesetting Unicode font tables}

When I worked on the font chapter for the new edition of \emph{The
  \LaTeX{} Companion, ed\,3}~\cite{TLC3} I wanted to produce glyph
tables for various fonts to examine which characters they encode and
how they looked. To my surprise I could not find any \TeX{} tool to do
this for me. There is, of course, the old \texttt{nfssfont} which I
had adapted from work by Don Knuth, but that is of no help with
Unicode fonts as it can only display tables of the first 256
characters, i.e. 8-bit fonts. So during my last stay at Bachotek
(before the pandemic) I sketched out some code, the result of which is
now available as the \pkg{unicodefonttable} package.


\fontsize{9.5}{11}\selectfont
\SetBibJustification{\raggedright}
\SetBibJustification{\raggedright \advance\itemsep by 2pt plus2pt minus1pt }

\bibliography{texbook1,\jobname}

\bibliographystyle{tugboat}


\makesignature


\end{document}

